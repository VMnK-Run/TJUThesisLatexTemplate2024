% !Mode:: "TeX:UTF-8"

%%  可通过增加或减少 setup/format.tex中的
%%  第274行 \setlength{\@title@width}{8cm}中 8cm 这个参数来 控制封面中下划线的长度。

\cheading{天津大学~2024~届本科生毕业论文}      % 设置正文的页眉,需要填上对应的毕业年份
\ctitle{春江花月夜}    % 封面用论文标题,自己可手动断行
\caffil{智能与计算学部} % 学院名称
\csubject{软件工程}   % 专业名称
\cgrade{2020~级}            % 年级
\cauthor{张若虚}            % 学生姓名
\cnumber{zhangruoxu111}        % 学生学号
\csupervisor{自学成才}        % 导师姓名

\csign{\quad \quad \quad \quad \quad \quad \quad \quad}

\teachersign{\quad \quad \quad \quad \quad \quad \quad \quad}

% \cdate{\the\year~年~\the\month~月~\the\day~日}



\setcounter{page}{1}                           % 单独从 1 开始编页码
\pagenumbering{Roman}
\thispagestyle{plain}

\cabstract{
《春江花月夜》是唐代诗人张若虚创作的七言歌行。此诗沿用陈隋乐府旧题,运用富有生活气息的清丽之笔,以江为场景,以月为主体,描绘了一幅幽美邈远、惝恍迷离的春江月夜图,抒写了游子思妇真挚动人的离情别绪以及富有哲理意味的人生感慨,突破了梁陈宫体诗的狭小天地,表现了一种迥绝的宇宙意识,创造了一个深沉、寥廓、宁静的艺术境界。

全诗共三十六句,每四句一换韵,通篇融诗情、画意、哲理为一体,意境空明,想象奇特,语言自然隽永,韵律宛转悠扬,为历代文人墨客吟咏唱诵,被闻一多誉为“诗中的诗,顶峰上的顶峰”。
}

\noin
\ckeywords{春江花月夜,唐诗,张若虚,七言歌行,游子思乡,离情别绪,唐代艺术文学赏析}

\eabstract{
A Night of Flowers and Moonlight in Spring is a seven-character song line created by Zhang Ruoxu, a poet in Tang Dynasty. This poem follows the old title of Chen Sui's Yuefu, uses a fresh and beautiful brush full of life, takes the river as the scene, takes the moon as the main body, describes a beautiful and distant spring moonlight picture, expresses the sincere and moving parting feelings of the wandering son and the thinking woman and the philosophical feelings of life, breaks through the narrow world of Liang Chen's palace poetry, and expresses a unique cosmic consciousness. Create a deep, boundless, quiet artistic realm.

The poem is composed of thirty-six sentences, every four sentences change rhyme, and the whole poem, painting and philosophy are integrated into one. The artistic conception is clear, the imagination is strange, the language is natural and meaningful, and the rhythm is melodious. It is recited and recited by scholars of all ages, and is praised by Wen Yiduo as "the poem in the poem, the peak on the peak".
}

\ekeywords{Keyword 1, Keyword 2, Keyword 3, Keyword 1, Keyword 2, Keyword 3, Keyword 1, Keyword 2, Keyword 3, }

\makecover

\clearpage
